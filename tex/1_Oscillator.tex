 \baselineskip14pt \parskip4pt plus2pt
 \def\hI{\hbox{I}} 
 \def\cc{\hbox{c.c.}}
 \def\cC{{\cal C}}  
 \def\cL{{\cal L}}
 \def\tx{\tilde x}
 \def\tJ{\tilde J} 
 \def\tI{\tilde I}  
 \def\tV{\tilde V}
 \def\xi{x_{rm i}}  
 \def\vi{v_{\rm i}}
 
\centerline{\large\textbf I.~Simple linear oscillators}\bigskip
 
\leftline{\bf A.  Definitions and Background}

Oscillators are a basic element in any measurement system and, in
particular, are the core element in any clocks.  By definition, an
{\it oscillator} is a device that produces that changes sinusoidally
as a function of time, so that if $s$ is the signal and $t$ is time, we
have $s(t) = s_0\cos(\omega t + \phi_0)$, where $s_0$ and $\phi_0$ are
constants, and $\omega=f/2\pi$ is the angular frequency and $f$ is the
usual frequency, measured in Hz or s$^{-1}$.  The signal can be a mechanical 
signal.  Examples are the oscillations of a pendulum or a tuning fork.  
Typical frequencies for a pendulum clock are on the order of 1 Hz, while
typical frequencies for a tuning fork are on the order of 500 Hz.
The signal can be an electrical signal.  Typical frequencies in this case
range from a few kilohertz to hundreds of MHz.  The signals can come from 
crystal vibrations.  Quartz crystal oscillators, which are the workhorse 
of most timekeeping systems, oscillate at frequencies from a few kHz to
tens of MHz.  The best oscillators use atomic transitions.  When an atom
or molecule gives up energy, it emits a pure electromagnetic wave.  The
oscillations in this case are at much higher frequencies.  For cesium
atoms, which are the basis of modern-day atomic clocks, the frequency
is close to $10\times10^9$ Hz or 10 GHz.  Optical transitions occur at
frequencies that are up to 100,000 times higher.  Clocks are currently
being developed at NIST that use a lattice of Yb atoms, that have an
oscillation frequency of about $500\times10^{12}$ Hz or 500 THz.

Operation at higher frequency is important because clocks are made by
counting the number of zero crossings.  A larger number of zero-crossings
per second makes it possible to develop more accurate clocks.

Strictly speaking, an oscillator does not have to produce a sinusoidal
signal.  It is sufficient for the signal to repeat periodically, so that
$s(t+T) = s(t)$, and in fact real oscillators do not produce strictly
sinusoidal signals.  Some {\it nonlinearity} is always present.  However, it 
is a remarkable fact of nature that any oscillator, when it operates at a
sufficiently low amplitude, will become sinusoidal.  Moreover, any real
oscillator will not operate with an absolutely constant amplitude $s_0$
and phase $\phi_0$.  There are always noise sources that make these
quantities {\it fluctuate} (vary around its average value) or {\it drift}
(walk away from its initial value).  We will characterize these effects
later.

 \bigskip
 \leftline{\bf B.~Second-Order Linear Oscillators}
 
The simplest model of a mechanical oscillator is a mass on a spring 
oscillating around a fixed equilibrium value $x_0=0$, as shown in Fig.~1.a.  
The force that the mass $m$ experiences is given by $F=-kx$, where $x$ is
the distance away from the equilibrium value and $k$ is the spring constant.
In this case, Newton's force law tells us that $F=ma$, where $a$ is the
acceleration or 
 $$m{d^2 x\over dt^2} = -kx.  \eqno(\hI.1)$$
Equation I.1 is a second-order linear differential equation with constant
coefficients.
It will be useful in the future to rewrite this equation as two first-order
equations.  If we define the velocity $v=dx/dt$, we find that Eq.~(I.1)
becomes
 $${dx\over dt} = v, \quad {dv\over dt} = -{k\over m}x.
    \eqno(\hI.2)$$
The solution to Eq.~(I.2) can be written in a number of different ways,
each of which has its uses.  One way to write the solution is
 $$x(t) = x_0\cos(\omega t + \phi_0),  \quad v(t) = -\omega 
    x_0\sin(\omega t + \phi_0),  \eqno(\hI.3)$$
where $\omega = (k/m)^{1/2}$.
The constants $x_0$ and $\phi_0$ are determined by the 
{\it initial conditions}
of the oscillator, or, in other words, the state that the oscillator is in
at the point in time that we pick to be the origin, $t=0$.  We see that 
$\xi\equiv x(t=0)=x_0\cos\phi$ and $\vi\equiv v(t=0)=-\omega x_0\sin\phi$, 
so that $a=(\xi^2 + \vi^2/\omega^2)^{1/2}$ and 
$\phi_0 = \tan^{-1}(-\vi/\omega \xi)$.  Another way to write the solution 
is
 $$x(t) = x_c\cos\omega t - x_s\sin\omega t, \quad
    v(t) = -\omega x_c\sin\omega t - \omega x_s\cos\omega t,  \eqno(\hI.4)$$
where $x_c=a\cos\phi$ and $x_s=a\sin\phi$.  A third way to write to write 
this solution is 
 $$x(t) = {1\over2}\tx\exp(i\omega t) + {1\over2}\tx^*\exp(-i\omega t), 
    \quad
    v(t) = {i\over2}\omega \tx\exp(i\omega t) 
    - {i\over2}\omega \tx^*\exp(-i\omega t).
    \eqno(\hI.5)$$
where $i\equiv\sqrt{-1}$.  The constant $\tx$ is a complex number, so that
$\tx=\tx_1 + i\tx_2$, where both $\tx_1$ and $\tx_2$ are real.  We have 
$\tx^* = \tx_1 - i\tx_2$ is the complex conjugate of $\tx$, and we note that
$\exp(\pm i\omega t) = \cos\omega t \pm i\sin\omega t$.  It follows that
$\tx_1=x_c$ and $\tx_2 = x_s$, so that $\tx=x_c+ix_s=x_0\exp(i\phi_0)$.  
It might seem as 
though the use of complex numbers unnecessarily complicates things, but, in 
fact, their use greatly simplifies the mathematical discussions.  The phase 
of the complex number $C$ corresponds to the phase offset in the real
displacement $x(t)$.

A point on notation:  Engineers typically use $j=\sqrt{-1}$, instead of $i$.
Mathematicians and physicists typically use $i$.  In optical engineering,
both notations can be found.

The simplest model of an electrical oscillator is an LC circuit, shown in
Fig.~1.b.  In this case, we find that
 $${d V(t)\over d t} = {1\over C}I(t), \quad {d I(t)\over dt} 
    = -{1\over L} V(t),  \eqno(\hI.6)$$
where $L$ is the inductance and $C$ is the capacitance.  This equation
is once again a linear second order differential equation.  Mathematically,
it is identical to the mechanical oscillator, even though the physical
system is completely different.  Same equations; same solutions!  So, we
can just write down the solution,
 $$V(t) = {1\over2}\tV\exp(i\omega t) + \hbox{c.c.}, 
 \quad I(t) = {i\over2Z_0} \tV\exp(i\omega t) + \hbox{c.c}, 
 \eqno(\hI.7)$$
where $\hbox{c.c.}$ is short for complex conjugate, 
$\omega = 1/(LC)^{1/2}$, $Z_0 = (L/C)^{1/2}$ is the characteristic
impedance of the circuit, and $\tV=\tV_1 + i\tV_2$ is determined by
the initial conditions.  We have $\tV_1=V(t=0)$ and $\tV_2=-I(t=0)Z_0$.

 \bigskip
 \leftline{\bf C.~Visualizing the Solutions}
 
In order to understand these solutions, it is useful to plot them.  We
will use {\tt MATLAB} for this purpose.  {\tt MATLAB} is a high-level
programming language that has embedded in it many special functions and
routines that are necessary for engineering and physics.  It also has
numerous routines for plotting.  UMBC has a license; so, it is available to 
all UMBC students for free.

When plotting solutions, it is necessary to pick a {\it normalization} for
the quantities since computers work with non-dimensional quantities.  It 
is possible to normalize with respect to the standard SI (Syst\`eme
International) units.  The SI units are the basic units in terms of which
all other units are defined [s = second, kg = kilogram, m = meter, A =
ampere, K = kelvin, mol = mole, cd = candela].  However, it is usually
However, it is usually best to work in units that
correspond to the physical system.  In normalizing time, we are typically
interested in times that are short compared to a second; so, we might
use units of ms (milliseconds, 10$^{-3}$ s), $\mu$s (microseconds, 
10$^{-6}$ s), ns (nanoseconds, 10$^{-9}$ s), or ps (picoseconds, 
10$^{-12}$ s).  In normalizing frequencies, we are typically interested in 
frequencies that are bigger than 1 Hz; so, we might use units of kHz 
(kilohertz, 10$^3$ Hz), MHz (megahertz, 10$^6$ Hz), GHz (gigahertz, 10$^9$
Hz), or THz (terahertz, 10$^{12}$ Hz). 

In a typical laboratory experiment with a mass on a spring, the mass
might equal 10 grams or $10^{-2}$ kg in standard SI units.
The spring might have a spring constant of 10 N/m.  In this case, the
radial frequency is given by $\omega = [10/10^{-2}]^{1/2} = 31.3$ s$^{-1}$,
so the frequency $f=\omega/2\pi = 5.03$ Hz, which means that the weight
will oscillate about five times per second.  So, normalizing with respect
to the second makes sense.  A typical maximum length $x_0$ might be on the
order of 1 cm (10$^{-2}$ m); so, normalizing lengths with respect 1 cm
makes sense.  In this case, we find that the maximum velocity is given
by $\omega x_0 = 0.316$ m/s.  

If we consider a typical 
laboratory experiment with an LC circuit, the capacitor might have a
capacitance of 100 pF (10$^{-10}$ F = 10$^{-10}$ 
s$^4$A$^2$kg$^{-1}$m$^{-2}$) and
the inductor might have an inductance of 1 mH (10$^{-3}$ H = 10$^{-3}$ 
s$^{-2}$A$^{-2}$kg m$^{2}$).  In this case, we find that $\omega =
3.16\times10^6$ s$^{-1}$, so that $f=503$ kHz or $f=0.503$ MHz.
So, it makes sense to work in units of kHz or perhaps MHz.  In 10 $\mu$s,
there are about five oscillations.  If the
amplitude of the voltage is 1 V, then the characteristic impedance
is given by $Z_0 = (10^{-3}/10^{-10})^{1/2}$ ohms ${}= 3.16\times10^3$
$\Omega$ = 3.16 k$\Omega$.  The amplitude of the current is then given
by $I_0 = 0.316$ mA or 316 $\mu$A\null.  

Figure 2.a shows the plot of this case as a function of time for five 
periods.  Figure 2.b shows phase plots in which both the current and voltage
are plotted as time varies.  These plots are useful as they show
qualitative features of the evolution that time plots don't reveal, and we
will use them many times.  We see that the phase plots close in on
themselves, which corresponds to the pattern repeating periodically,
as is required for a good oscillator.  Here, we show three different
cases, corresponding to $V_0 = 0.5$, 0.7, and 0.9.  We also show the
MATLAB code that we used to generate the plots.  The curves are elliptical
in shape because the oscillator energy is constant.  
The energy of the mechanical oscillator is given by
 $$U = {1\over 2} mv^2 + {1\over2} k x^2,  \eqno(\hI.8)$$
and the energy of the electrical oscillator is given by
 $$U={1\over2}LI^2 +{1\over 2C}V^2.  \eqno(\hI.9)$$

 \bigskip
 \leftline{\bf D. Damped-Driven Oscillators}
 
In reality, all physical systems are damped.  In the case of the mechanical
oscillators, our equations become
 $${dx\over dt} = v, \quad {dv\over dt} = -\alpha v -{k\over m}x,
    \eqno(\hI.10)$$
For the electrical oscillator, if we consider the RLC circuit that we
show in Fig.~3, we find
 $${dV\over dt} = {1\over C}I, \quad {dI\over dt} = -{1\over RC}I
    -{1\over L}V,  \eqno(\hI.11)$$
where $R$ is the resistance.  Loss in this circuit is reduced when the
resistance is large so that there is little current flow through the
resistor.  Focusing on the electrical oscillator, we
find that Eq.~I.11 has the general solution
 $$V(t) = {1\over2}\tV_+\exp(\lambda_+ t) 
    + {1\over2}\tV_- \exp(\lambda_- t), 
    \quad I(t) = {\lambda_+\over 2} \tV_+\exp(\lambda_+ t) 
    + {\lambda_- \over 2} \tV_- \exp(\lambda_- t),  \eqno(\hI.12)$$
where
 $$ \lambda_{\pm} = -{1\over 2 RC} \pm \left[\left({1\over 2RC}\right)^2
    -{1\over LC}\right]^{1/2}.  \eqno(\hI.13) $$  
We see that if $(1/2RC)^2 > 1/LC$, then both $\lambda_+$ and $\lambda_-$
are real, while if $1/LC > (1/2RC)^2$, then $\lambda_+$ and $\lambda_-$
are complex conjugate numbers.  For oscillators, we are interested in the
second case, where the damping is small.  In this case, we find
$\lambda_+ = i\omega - \gamma$, where
 $$\omega = \left[{1\over LC} - \left(1\over 2RC\right)^2\right]^{1/2}
    = {1\over(LC)^{1/2}}\left[1 -{1\over 8}{L\over RC^2} +   
    \cdots\right]^{1/2}
    \simeq {1\over (LC)^{1/2}},  \eqno(\hI.14)$$
where we have written a Taylor expansion for $\omega$ and just kept the
lowest-order term.  We find $\gamma = 1/2RC$.  In this limit, we may
write
 $$V(t) = {1\over 2}\tV\exp[(i\omega - \gamma)t] + \cc, I(t) 
    = {i\over 2Z_0}\tV\exp[(i\omega - \gamma)t] + \cc,  \eqno(\hI.15)$$
where $1/Z_0 = (\omega + i\gamma)C$.  Hence,
the voltage and current are no longer exactly $\pi/2$ out of phase.
Writing the voltage and current explicitly as real quantities, we find
\begin{align} 
V(t) &= V_0\cos(\omega t + \phi_0)\exp(-\gamma t), \tag{I.16} \\
    I(t) &= -{V_0\over (L/C)^{1/2}}\sin(\omega t + \phi_0 
       + \phi_{\rm off}), 
\end{align}
where $V_0$ and $\phi_0$ are two constants determined by the initial
conditions, and $\phi_{\rm off} = \tan^{-1}(\gamma/\omega)$

Returning to the oscillator circuit that we considered in Sec.~I.C and
setting the resistance $R$ equal to 50 k$\Omega$, $V_0 = 1$ V, and
$\phi_0 = 0$, we show the evolution
in Fig.~4.a.  The resistance causes the voltage and current to spiral
into the origin, ultimately damping away completely.

This ``death spiral'' is of course unacceptable in real oscillators, and
it must be compensated by a source of energy that compensates for the
loss.  So, for example, in the RLC circuit that we are considering, we
can add a current source, as shown in Fig.~5.  Writing the time
derivative of the current source $J(t)$ as $J'(t)$, we find that the
equations that govern the voltage and current become
 $${dV\over dt} - {1\over C}I = 0, \quad {dI\over dt} + {1\over RC}I
    +{1\over L}V = J',  \eqno(\hI.17)$$
    
We begin by considering the case where the driver is a pure oscillating
signal, so that
 $$J=J_0\cos(\omega t + \phi_0) = {1\over2}\tJ\exp(i\omega_0 t) + \cc,
    \eqno(\hI.18)$$
where $J_0$ and $\phi_0$ are constants, and $\tJ=J_0\exp(i\phi_0)$
is a complex constant.  We will denote the resonant frequency of the
oscillator as $\omega_0$, so that
 $$\omega_0 = \left({1\over LC} - \gamma^2\right)^{1/2} \simeq {1\over(LC)^{1/2}}.
    \eqno(\hI.19)$$
It is useful to choose the time origin so that $\phi_0 = 0$, in which
case $\tJ$ is purely real.  Because our equations are linear, we can
find the solutions for $V(t)$ and $I(t)$ by just setting $J(t)=\tJ
\exp(i\omega t)$, so that $J' = i\omega\tJ
\exp(i\omega_0 t)$ and then then taking the real part at the end.
That turns out to be the simplest approach mathematically.  There will
be an initial transient that damps out, after which both $I(t)$ and $V(t)$
will be proportional to $\exp(i\omega t)$.  Writing $I(t) = \tI
\exp(i\omega_0t)$ and $V(t) = \tV\exp(i\omega t)$, we find
 
 \bigskip
 \leftline{\bf E. Matrix Representations:  Eigenvalues and Eigenvectors}
 
 \bigskip
 \centerline{\bf Exercises}
 



 \vfil\eject
